\documentclass[a4paper,12pt]{article}
\usepackage{amsmath,amsthm,amssymb}
\begin{document}

5. Prove that for any integer $n$, at least one of the integers $n$, $n + 2$, $n
+ 4$ is divisible by 3.

Pre-proof note:

$n + 4$ is equivalent to $n + 1 + 3$. This is clearly divisible by 3 if, and
only if, $n + 1$ is divisible by three. We will therefore consider $n + 1$
instead of $n + 4$ with no loss of generality.

\emph{Theorem}: For all integers $n$ at least one of the following expressions is divisible by 3
\begin{enumerate}
	\item $n$
	\item $n + 1$
	\item $n + 2$
\end{enumerate}

\begin{proof}

For any integer $n$, $n$ is either divisible by 3 (case $1.$) or there is some remainder
$r$ (where $r$ = 1 or $r$ = 2).
If $r = 2$, we can add 1 to $n$ to get a number divisible by 3 (case $2.$).
If $r = 1$, we can add 2 to $n$ to get a number divisible by 3 (case $3.$).

This proves that the theorem is true, for any integer $n$ one of the three cases
stated above is divisible by 3.

\end{proof}

\end{document}
