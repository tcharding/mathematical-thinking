\documentclass[a4paper,12pt]{article}
\usepackage{amsmath,amsthm,amssymb}
\begin{document}

2. Say whether the following is true or false and support your answer by a proof: 
The sum of any five consecutive integers is divisible by 5 (without remainder).

\begin{proof}

  By mathematical induction.

  For any integer $x$ the sum ($X$) of the first 5 consecutive integers starting at
  $x$ can be expressed as:

  $$X = x + (x + 1) + (x + 2) + (x + 3) + (x + 4)$$

  For $x = 0$, we have:

\begin{align*} 
  X &= 0 + (0 + 1) + (0 + 2) + (0 + 3) + (0 + 4) \\
    &= 1 + 2 + 3 + 4 \\
    &= 10 \\
    &= 5(2) \\
    &= 5n \tag{divisible by five}
\end{align*}

This is the inductive base case. To prove this relation for all $x$ we show that
it is valid for both positive and negative $x$.

For positive $x$:

\begin{align*} 
  X &= x + (x + 1) + (x + 2) + (x + 3) + (x + 4) \\
    &= (x + 1) + ((x + 1) + 1) + ((x + 1) + 2) + ((x + 1) + 3) + ((x + 1) + 4) 
  \tag{subsitute x+1} \\
    &= 5 + (x + (x + 1) + (x + 2) + (x + 3) + (x + 4)) 
  \tag{bring plus ones out the front} \\
    &= 5 + 5n \\ \tag{from the inductive statement} \\
    &= 5(1 + n)
\end{align*}

This proves that the inductive statement is true for as $x$ tends to $\infty$.

For negative $x$:

\begin{align*} 
  X &= x + (x - 1) + (x - 2) + (x - 3) + (x - 4) \\
    &= (x - 1) + ((x - 1) + 1) + ((x - 1) + 2) + ((x - 1) + 3) + ((x - 1) + 4) 
  \tag{subsitute x+1} \\
    &= -5 + (x + (x + 1) + (x + 2) + (x + 3) + (x + 4)) 
  \tag{bring plus ones out the front} \\
    &= -5 + 5n \\ \tag{from the inductive statement} \\
    &= 5(-1 + n)
\end{align*}

This proves that the inductive statement is true for as $x$ tends to $-\infty$,
thereby proving the theorem is true for all $x$ (by mathematical induction).
  
\end{proof}

\end{document}
